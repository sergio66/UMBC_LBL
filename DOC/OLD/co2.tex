%\documentclass[11pt]{elsart}
\documentclass[11pt]{article}

\usepackage{longtable}
%\usepackage{lucbr}
\usepackage{graphicx}
\usepackage{color}
%\usepackage{strow_defs}

\input strowpreprint

%\input ASL_defs

\begin{document}

%\begin{frontmatter}

\title{MatlabLBLco2 : An Algorithm to compute CO2 Line-by-Line spectra}

\author{Sergio De Souza-Machado, L. Larrabee Strow}
\author{David Tobin, Howard E. Motteler, and Scott E. Hannon} 

%\address{University of Maryland Baltimore County, Baltimore, MD 21250 USA}

\begin{abstract}
  This file tells the user how to modify $run6co2.m$ and its associated files 
  so that one can vary the bands used, duration of collision and other
  parameters.

\end{abstract}

%\end{frontmatter}

\section{run6co2 : choosing the band}
Basically, the code first reads in all lines from the HITRAN database, above
a certain minimum strength and between the specified start/stop wavenumbers, 
$modulo$ parameter xfar (typically $\pm 150 cm^{-1}$ for CO2).

These lines can then be broadly broken into two categories : those which belong
to bands for which there exists line mixing code, and those for which there
is no linemixing code. We call these two sets of lines as those belonging to
``bands'' and those belonging to the ``background''.

At present, the code defaults to computing line mixing for all sigpi, sigsig, 
deltpi, deldelt and pipi bands that it can handle. These are all specified
in the section  titled ``REMOVE THE BAND DATA FROM THE BACKGROUND''. This 
section separates the ``band'' lines from the rest of the ``background'' 
lines, so that the lines are not used twice in the computation of the spectra.
If the user wants to limit the bands for which line mixing is computed,
this section has to be modified.

\begin{verbatim}
%%%%%%%%%%%%%%% REMOVE THE BAND DATA FROM THE BACKGROUND %%%%%%%%%%%%%%% 
CO2q_sigpi = [618 648 667 720 791 2080]; 
CO2q_delpi = [668 740 2093]; 
bandQ=union(CO2q_sigpi,CO2q_delpi); 
 
CO2pr_sigsig   = [2350 2351 2352 2353 2354]; 
CO2pr_deltdelt = [2310 2311]; 
CO2pr_pipi     = [2320 2321 2322]; 
CO2pr_sigpi    = [667 720]; 
bandPR=union(CO2pr_sigsig,CO2pr_deltdelt); 
bandPR=union(bandPR,CO2pr_pipi); 
bandPR=union(bandPR,CO2pr_sigpi); 
 
band=union(bandQ,bandPR); 
 
%ZZZ 
%%%%%band=[2350];  
\end{verbatim}

If the user wanted to only include the main sigsig band (2350), he/she would 
search for $ZZZ$ and then change band to band=[2350].\\
If the user wanted to only include all the sigsig bands, he/she would 
search for $ZZZ$ and then change band to band=[2350 2351 2352 2353 2354].\\

\section{run6co2 : turning off background lines}
Having chosen what band(s) to use, the user can now choose to either
turn off all remaining background lines, or to use all remaining
background lines. At present the code defaults to using all remaining 
background lines. To turn this feature off, do another search for $ZZZ$ : you 
should end up in the following section

\begin{verbatim}
%%%%%%%%%%%%%%%%%%%%%% do background calc  %%%%%%%%%%%%%%%%%%%%%%%%%%%%% 
domesh = +1;   %this does stuff the fast way!!!!!!! ie Genln2 way 
domesh = -1;   %this does stuff the slow way!!!!!!! add ALL lines everywhere 
 
domesh = +1;   %this does stuff the fast way!!!!!!! ie Genln2 way 
 
if domesh > 0 
 fprintf(1,'\n doing BACKGROUND ... \n'); 
%%%%%%%ZZZ 
 for  ii=1:nwide    %OUTER LOOP OVER WIDE MESH  
%%  for  ii=1:-1    %OUTER LOOP OVER WIDE MESH  
\end{verbatim}

If the user wants to turn off the background lines, comment out the first loop
and uncomment the second loop (which, as you will notice, does NOTHING as it
is not executed!!!)
\begin{verbatim}
%% for  ii=1:nwide    %OUTER LOOP OVER WIDE MESH  
  for  ii=1:-1    %OUTER LOOP OVER WIDE MESH  
\end{verbatim}

\section{removeCO2lines : specifying P vs R}
Using $band$ in $run6co2$ as specified above, only allows the user to turn 
on/off a complete band (P and R (and mebbe Q)). If the user wants finer control
over this, then he/she must go to file $removeCO2lines.m$.
The $PQR$ lines for a band are coded according to 

\begin{verbatim}
%PQR is a code that tells us which bands to add in  01 = Q delt pi 
%                                                   02 = Q sig  pi 
 
%                                                  -11 = P sig  sig 
%                                                  -12 = P delt delt 
%                                                  -13 = P pi   pi 
%                                                  -14 = P sig  pi 
%                                                  -15 = P delt pi 
 
%                                                  +11 = R sig  sig 
%                                                  +12 = R delt delt 
%                                                  +13 = R pi   pi 
%                                                  +14 = R sig  pi 
%                                                  +15 = R delt pi 
%bandtype tells you which band to do eg 740 etc 
\end{verbatim}

By default, all the above codes are searched for, by specifying them all
in an array $IndicesToSearch$. To turn on/off some branches, the user 
must reset the $IndicesToSearch$ array, as well as go down the file and turn 
off all unnecessary branches.
\begin{bf}
this is quite tricky, and I would not recommend messing with this part of 
the code unless you know what you are doing!!!!!!!
\end{bf}

Note that at present the code has the default of using ALL branches for 
ALL bands :
\begin{verbatim}
ugh=-1; %do not do the PR 15 um band line mixing 
ugh=0;  %only do Q720,741 bands 
ugh=+1; %do everything  

ugh=+1;
\end{verbatim}
If the user only wanted to do two Q branches in 15 um band (Q720,740), then 
$ugh$ should be reset to 0. 

However, the user can control the branches even more delicately.
For example, suppose the user only wanted to use the R sig sig branch.
In $run6co2.m$ he/she would have selected $band=[2350]$, but he/she can now
further tune this choice by uncommenting the second line below, and commenting 
out the third line: 

\begin{verbatim}
  %this would only do the R sig sig branch!!!!!! 
  %IndicesToSearch=[01 02     11 -12 12 -13 13 -14 14 -15 15]; 
   IndicesToSearch=[01 02 -11 11 -12 12 -13 13 -14 14 -15 15]; 
\end{verbatim}

One sees that $-11$ is missing from the array. The user now has to search for
$P_sigsig$, and comment out all lines in that block
\begin{verbatim}
%  iii = iii+1; 
%  P_sigsig        =  [2350 2351 2352 2353 2354]; 
%  P_sigsig_lower  =  [1    1    1     3    5]; 
%  P_sigsig_upper  =  [9    9    9     23   25]; 
%  P_isotope      =   [1    2    3     1    1]; 
%  str.name(iii,1:10) = 'P_sigsig  ';  
%  str.ID(iii,1)      = -11;    
%  str.PQRtype(iii,1) = 'P'; 
%  str.number(iii,1)  = 5; 
%  str.iso(iii,1:str.number(iii,1))      = P_isotope; 
%  str.bbb(iii,1:str.number(iii,1))      = P_sigsig;        
%  str.thelower(iii,1:str.number(iii,1)) = P_sigsig_lower;  
%  str.theupper(iii,1:str.number(iii,1)) = P_sigsig_upper; 
\end{verbatim}
without touching anything else!!!!!!!!!!!!!!!

\section{co2param.m}
This file,in subdir CO2\_COMMON, allows the user to globally control the 
beta and duration of collision parameters, for ALL bands. 

For the Q delta\_pi band, we need four parameters : beta\_pi\_self, 
beta\_pi\_air, beta\_delt\_self,beta\_delt\_air, so these are sent out through 
beta\_self,beta\_for,b2s,b2f, with the first two parameters (ie the 
duration of collision) being irrelevant.

For the PR delta\_pi band, we only need the duration of collision parameter,
as all line mixing is computed using $k/klor = 0.5$. Thus the first two
parameters sent out are important, the rest are irrelevant.

For all other bands, we need the two duration of collision parameters, and the 
the two beta parameters. Thus the last two parameters that are sent out are 
irrelevant (and are set to 0.0)

The duration of collision parameters were computed as follows. Using Dave 
Tobin's PR\_sigsig GLOBAL fitting code, files 8,9,10 were used to determine
duration, but since these were for for pure CO2, this is a measurement of
duration\_self = 9.244381131571670e-03. Files 2,3,4,5,6 were used to compute 
the duration factor for an air/Co2 mix, obtaining a value of 
duration = 3.872568524276932e-03; assuming \\
\begin{math}
duration=(pself*duration\_pure+(ptotal-pself)*duration\_for)/ptotal;  
\end{math}
we get duration\_for = 3.590130719741353e-03;

For the 4 um band, the beta values were obtained in a very similar fashion,
yielding beta\_self = 7.178170098503445e-01, beta  = 9.815991792047214e-01,
beta\_for=9.954682513847358e-01

\begin{verbatim}
%this file sets the beta parameters etc
function [duration_self,duration_for,beta_self,beta_for,b2s,b2f] = ...
         co2_param(band)
%function [duration_self,duration_for,beta_self,beta_for,b2s,b2f] = ...
%         co2_param(band)


%b2s,b2f=0 for all bands except the PR_deltpi bands
b2s=0.0;
b2f=0.0;

%use this d.of.c parameter everywhere!!!!!!!
%this is from the files 8,9,10 in PR_sigsig/GLOBAL
duration_self = 9.244381131571670e-03;

%this is from the files 2,3,4 in PR_sigsig/GLOBAL
duration  = 3.872568524276932e-03;
%duration=(pself*duration_pure+(ptotal-pself)*duration_for)/ptotal; 
%for files 2,3,4 ptotal= 1.002026315789474, pself=0.05005263157894737
%%%%%%%%%%%%was using duration_for = 3.589841544945630e-03  ptotal=1.002
duration_for = 3.590130719741353e-03;

if ((band == 618)|(band == 648)&(band == 662)|(band == 667)|(band == 720)|...
              (band == 791)|(band == 2080))
  %this is for the 15 um band
  %sigpi band   
  %for Q branch, duration is irrelevant, so this is used by PR_sigpi/loader.m
  %used for PQR branches
  beta_self = 0.5358;            %self broadened        %%%%%used to be 0.5
  beta_for  = 0.6002;            %foreign broadened     %%%%%used to be 0.62
  
elseif  ((band == 668)|(band == 740)|(band ==2093))
  %this is for the 15 um band
  %deltpi band    for Q branch, duration is irrelevant
  beta_pi_self   = 0.5407505;  % fitted from pure Pi-Sigma data 
  beta_pi_air    = 0.5995593;  % fitted from air-broadened Pi-Sigma data 
  beta_delt_self = 0.4880067;  % fitted from pure Pi-Delta data 
  beta_delt_air  = 0.72172644; % fitted from air-broadened Pi-Delta data 

  beta_self     = beta_pi_self;
  beta_for      = beta_pi_air;
  b2s           = beta_delt_self;
  b2f           = beta_delt_air;

  %for PR branch, we DO NOT do complete line mixing ie we just use k/klor=0.5
  %and then use the d. of .c parameter there
  %so this file is called by PR_deltpi but just uses d. of. c
  %this is for the 4 um band
elseif  ((band == 2310)|(band == 2311)  |  ...                      %deltdelt
         (band == 2320)|(band == 2321)|(band == 2322) | ...         %pipi
         (band== 2350)|(band== 2351)|(band == 2352)|(band ==2353)| ...
         (band==2354))                                              %isgsig
  %this is from the files 8,9,10 in PR_sigsig/GLOBAL
  beta_self = 7.178170098503445e-01; 
  %this is from the files 2,3,4 in PR_sigsig/GLOBAL
  beta  = 9.815991792047214e-01; 
  %beta=(pself*beta_pure+(ptotal-pself)*beta_for)/ptotal; 
  %for files 2,3,4 ptotal= 1.002026315789474, pself=0.05005263157894737
  %%%%%%% was using beta_for=9.954824512760044e-01 (ptotal=1.002)
  beta_for=9.954682513847358e-01;

else
  error('none of bands in this file matches YOUR band!!')
  end


\end{verbatim}

\end{document}